\chapter{Điều khiển và sao chép dữ liệu với Raspberry Pi}
%Có một cách có thể dùng để điều khiển Raspberry Pi
\section{Điều khiển bằng cách kết nối trực tiếp với màn hình, bàn phím và chuột}
\begin{itemize}
\item \textit{Điều khiển}: Kết nối chuột và bàn phím qua các cổng USB. Với màn hình thông thường có 2 loại: màn hình hỗ trợ cổng HDMI và màn hình hổ trợ cổng VGA.
\item \textit{Sao chép dữ liệu}: Sử dụng USB.
\end{itemize}
\subsection{Màn hình hổ trợ cổng HDMI}
Ta kết nối màn hình qua cable HDMI. Có thể bạn sẽ cần tùy chỉnh một số thông sau cho phù hợp:
\subsection{Màn hình hổ trợ cổng VGA}
Để hiển thị được, ta cần có cable chuyển đồi từ VGA sang HDMI.
\section{Điều khiển bằng giao tiếp nối tiếp thông qua cổng RS232}
\begin{itemize}
\item Thực hiện kết nối Pi và module RS232 như sau:
\begin{center}
\begin{tabular}{c|c}
Pi & RS232\\ \hline
3.3V & VCC\\
TX & TX \\ 
RX & RX\\
GND & GND
\end{tabular}
\end{center}
\item Cài đặt gói phần mềm \verb|screen|: \verb|sudo apt-get install screen| trên máy tính Ubuntu.
\item Chạy lệnh sau: \verb|sudo screen /dev/ttyUSB0 115200|
\item Thực hiện xong lệnh trên, ta nhấn Enter một lần nữa để kết nối với Pi.
\item Nhập username và password để đăng nhập.
\item Sao chép dữ liệu: dùng USB.
\item[$\ast$] Ta có thể dùng Putty (trên hệ điều hành Window) để điều khiển: chọn \verb|Serial|, điền vào khung \verb|Serial line| tên cổng (ví dụ: COM1, COM2,\ldots), trong khung \verb|Speed| điền tốc độ là \verb|115200|. Nhập username và password để đăng nhập.
\end{itemize}
\section{Điều khiển từ xa khi Raspberry Pi có kết nối mạng}
Khi Raspberry Pi có kết nối mạng Internet, ta có thể dùng các phần mềm: \verb|SSH|, \verb|Remote Desktop|, \verb|VNC|,\ldots~ để điều khiển.
\begin{itemize}
\item Kiểm tra địa chỉ IP của Pi bằng phần mềm: ipscan (trên Windows) hoặc nmap (trên Ubuntu).
\item Chọn chương trình phù hợp để điều khiển Raspberry Pi: 
\begin{itemize}
\item Với \verb|SSH|: không hổ trợ giao diện đồ họa.
\item Với \verb|Remote Desktop| (Pi cần cài đặt: \verb|xrdp|, dùng lệnh: \verb|sudo apt-get install xrdp|), \verb|VNC|: có hổ trợ giao diện đồ họa.
\end{itemize}
\item Tùy theo chương trình bạn chọn: ta cần phải nhập địa chỉ IP, username và password (nếu có yêu cầu điền số \verb|port|: ta điền 22).
\item Sao chép dữ liệu:
\begin{itemize}
\item Trên Window: dùng \verb|Winscp|.
\item Trên Ubuntu: dùng \verb|FileZilla|.
\item[$\ast$] Ta cũng cần nhập vào thông tin như trên để truy cập được Pi.
\end{itemize}
\item \textit{Lưu ý}: Phần trình bày trên áp ngay cho mạng nội bộ, khi không phải mạng nội bộ ta cần cấu hình mạng rồi mới áp dụng được hướng dẫn ở phần này.
\end{itemize}

